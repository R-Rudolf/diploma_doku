\chapter{Együttműködések}
%(6 oldal): Hogyan csatlakoztak be, mit értek el, mivel lett több az egész, együttműködések menete.

A mérési rendszer fejlesztése 2015-ben kezdődött önálló laboratóriumi munkaként. Az első működő verziót követően további témakiírások készültek hozzá kapcsolódóan. Ezek közreműködési lehetőséget biztosítottak a hallgatóknak, hogy saját munkájuk során a meglévő mérési rendszert vagy annak eredményeit felhasználják és kiegészítsék azt. A következőkben ezen munkák kerülnek bemutatásra. Az egyes leírások fontos eleme a mérési rendszerhez való hozzáadott értékek és a rendszer felhasznált szolgáltatásainak kiemelése.

Ezen együttműködéseknél rendkívül fontos volt az ismeret átadás, és a munkák koordinálása. A konzulensemmel, Dr Heszberger Zalánnal, végig gondot fordítottunk arra hogy minden szükséges tudást átadjunk a munkákhoz, és felügyeljük azok végrehajtását. Minden esetben elégedettek voltunk a közös munka gyümölcsével, és sokszor kellemesen meglepett milyen távolra vezettek az általunk elindított témák.

Az egyes fejezetcímek az elkészült dolgozatok címének felelnek meg.

\section{Forgalmi mérési környezet kialakítása az Interneten}
%Haja Dávid munkájának rövid bemutatása (2 oldal)

% Mit adott hozzá az én témámhoz?
% Az én témámnak mely részeire tudta építeni a saját munkáját?

Haja Dávid alapszakos hallgató a szakdolgozataként választotta az általunk kiírt témát. Munkája során megismerte az Elméleti Áttekintés fejezet tudásanyagát, valamint a mérőrendszer felépítését. Fő tevékenységei a PlanetLab hálózat komolyabb megismerése volt, valamint a mérési rendszer kiegészítése sávszélesség kapacitás méréssel.

\subsection*{PlanetLab rendszerének megismerése}
A mérések végrehajtása során a PlanetLab hálózat gépeinek elérésével gondjaink akadtak. A tanszéken üzemelő szerverek nem megfelelően voltak felkonfigurálva, nem üzemeltek rendeltetésszerűen. A kölcsönösség elve alapján így nem fértünk hozzá a PlanetLab hálózatának szervereihez.
Haja Dávid közreműködésével az említett szerverek helyre lettek állítva. Ezen munkája során mély ismeretekre tett szert a PalnetLab szervezet működésével és a szerverek felépítésével kapcsolatban.

\subsection*{D-ITG sávszélesség mérés}
A mérési rendszer eredetileg csak iperf sávszélesség mérésekkel rendelkezett, mivel ez a szoftver a legtöbb linux alapú rendszereken, így a PlanetLab gépein is előre telepítve volt. A mérések színvonalának érdekében azonban Haja Dávid feladata volt felmérni mely további sávszélességmérő rendszerek nyújthatnak megbízható és részletes információkat. Kutatásai során választása a D-ITG-re (Distributed Internet Traffic Generator) esett. Ez a mérőszoftver bír az elvárható alapképességekkel, mint a kinyerhető adatok széles választéka, protokoll támogatások, párhuzamos folyamok kezelése, valamint sztochasztikus adatküldő profilok is beállíthatóak. A program azonban nem elérhető a publikus csomagkezelő rendszerekből, ahonnan egyszerűen telepíthetőek lennének. Emiatt munkájának nagy része volt a szoftver becsomagolása a PlantLab gépeinek megfelelő RPM formátumba. 

Automatizált telepítés Ansible-lel


Ide kép a D-ITG mérés eredményéről

\section{AS útvonalváltozások vizsgálata}
Kocsmár Bence munkájának rövid bemutatása (2 oldal)

\section{Adatbázis szervezés webes környezetben}
Patkó Ákos munkájának rövid bemutatása (1-2 oldal)
















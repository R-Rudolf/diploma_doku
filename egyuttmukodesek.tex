\chapter{Együttműködések}
%(6 oldal): Hogyan csatlakoztak be, mit értek el, mivel lett több az egész, együttműködések menete.

A mérési rendszer fejlesztése 2015-ben kezdődött önálló laboratóriumi munkaként. Az első működő verziót követően további témakiírások készültek hozzá kapcsolódóan. Ezek közreműködési lehetőségeket biztosítottak hallgatóknak, hogy saját munkájuk a meglévő mérési rendszert vagy annak eredményeit felhasználják és kiegészítsék azt. A következőkben ezen munkák kerülnek bemutatásra. Az egyes bemutatások fontos eleme a mérési rendszerhez való hozzáadott értékek és a rendszer felhasznált szolgáltatásainak kiemelése.

Ezen együttműködések során rendkívül fontos volt az ismeret átadás, és a munkák koordinálása. A konzulensemmel, Dr Heszberger Zalánnal végig gondot fordítottunk arra, hogy minden szükséges tudást átadjunk a munkákhoz és felügyeljük azok végrehajtását. Minden esetben elégedettek voltunk a közös munka gyümölcsével, és sokszor rácsodálkoztunk milyen távolra vezettek az általunk elindított témák.

\section{Forgalmi mérési környezet kialakítása az Interneten}
Haja Dávid munkájának rövid bemutatása

\section{AS útvonalváltozások vizsgálata}
Kocsmár Bence munkájának rövid bemutatása

\section{Adatbázis szervezés webes környezetben}
Patkó Ákos munkájának rövid bemutatása
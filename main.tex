\documentclass[11pt, a4paper]{report}
%\documentclass[11pt,a4paper,oneside]{report}

%Packages
\usepackage[hungarian]{babel}
\usepackage[utf8]{inputenc}
\usepackage{graphicx}
%
\usepackage{hyperref}
\usepackage{wrapfig}
\usepackage{subfigure }
\usepackage[margin=80pt]{geometry}
\usepackage{geometry}
\usepackage{amsmath}
\usepackage{tabularx}
\usepackage{tabulary}
%\usepackage{url}
\usepackage{multicol}
\usepackage{multirow}
\usepackage{setspace}
\usepackage[table,xcdraw]{xcolor}
%\usepackage[acronym]{glossaries}
\usepackage{longtable}
\usepackage{titlesec}



\usepackage[procnames]{listings}
\usepackage{color}


\definecolor{keywords}{RGB}{255,0,90}
\definecolor{green}{RGB}{0,0,113}
\definecolor{red}{RGB}{160,0,0}
\definecolor{comments}{RGB}{0,150,0}
 
\lstset{
    frame=single,
    breaklines=true,
    postbreak=\raisebox{0ex}[0ex][0ex]{\ensuremath{\color{red}\hookrightarrow\space}}
}

\lstset{language=Python, 
        basicstyle=\ttfamily\small, 
        keywordstyle=\color{keywords},
        commentstyle=\color{comments},
        stringstyle=\color{red},
        showstringspaces=false,
        identifierstyle=\color{green},
        procnamekeys={def,class}}

\usepackage{color}
\definecolor{lightgray}{rgb}{.9,.9,.9}
\definecolor{darkgray}{rgb}{.4,.4,.4}
\definecolor{purple}{rgb}{0.65, 0.12, 0.82}
\lstdefinelanguage{JavaScript}{
  keywords={break, case, catch, continue, debugger, default, delete, do, else, false, finally, for, function, if, in, instanceof, new, null, return, switch, this, throw, true, try, typeof, var, void, while, with},
  morecomment=[l]{//},
  morecomment=[s]{/*}{*/},
  morestring=[b]',
  morestring=[b]",
  ndkeywords={class, export, boolean, throw, implements, import, this},
  keywordstyle=\color{blue}\bfseries,
  ndkeywordstyle=\color{darkgray}\bfseries,
  identifierstyle=\color{black},
  commentstyle=\color{purple}\ttfamily,
  stringstyle=\color{red}\ttfamily,
  sensitive=true
}

\lstset{
   language=JavaScript,
   backgroundcolor=\color{lightgray},
   extendedchars=true,
   basicstyle=\footnotesize\ttfamily,
   showstringspaces=false,
   showspaces=false,
   numbers=left,
   numberstyle=\footnotesize,
   numbersep=9pt,
   tabsize=2,
   breaklines=true,
   showtabs=false,
   captionpos=b
}
%
%\lstset{ %
%  language=R,                     % the language of the code
%  basicstyle=\footnotesize,       % the size of the fonts that are used for the code
%  numbers=left,                   % where to put the line-numbers
%  numberstyle=\tiny\color{gray},  % the style that is used for the line-numbers
%  stepnumber=1,                   % the step between two line-numbers. If it's 1, each line
%                                  % will be numbered
%  numbersep=5pt,                  % how far the line-numbers are from the code
%  backgroundcolor=\color{white},  % choose the background color. You must add \usepackage{color}
%  showspaces=false,               % show spaces adding particular underscores
%  showstringspaces=false,         % underline spaces within strings
%  showtabs=false,                 % show tabs within strings adding particular underscores
%  frame=single,                   % adds a frame around the code
%  rulecolor=\color{black},        % if not set, the frame-color may be changed on line-breaks within not-black text (e.g. commens (green here))
%  tabsize=2,                      % sets default tabsize to 2 spaces
%  captionpos=b,                   % sets the caption-position to bottom
%  breaklines=true,                % sets automatic line breaking
%  breakatwhitespace=false,        % sets if automatic breaks should only happen at whitespace
%  title=\lstname,                 % show the filename of files included with \lstinputlisting;
%                                  % also try caption instead of title
%  keywordstyle=\color{blue},      % keyword style
%  commentstyle=\color{dkgreen},   % comment style
%  stringstyle=\color{mauve},      % string literal style
%  escapeinside={\%*}{*)},         % if you want to add a comment within your code
%  morekeywords={*,...}            % if you want to add more keywords to the set
%} 

\pagestyle{plain}
\setlength{\parindent}{12pt} % magyar nyelvû dokumentumokban jellemzõ
\setlength{\parskip}{0pt}    % magyar nyelvû dokumentumokban jellemzõ
\frenchspacing
\setlength{\columnsep}{1cm}
\linespread{1.5}

\includeonly{
	cimlap,
	nyilatkozat,
	eloszo,
    internet,
    mereselmelet,
    felderites,
    meres,
    fejlesztes,
    eredmenyek,
    feldolgozas,
    egyuttmukodesek,
	zaras,
	fuggelek,
}
%%%%%%%%%%%%%%%%%%%%%%%%%%%%%%%%%%%%%%%%%


\begin{document}

%------------------------------------------------
% Eleje (4 oldal): tartalomjegyzék, abstract, bevezető, borító, stb...

%--------------------------------------------------------------------------------------
%	The title page
%--------------------------------------------------------------------------------------
\begin{titlepage}
\begin{center}
\includegraphics[width=60mm,keepaspectratio]{figures/BMElogo.png}\\
\vspace{0.3cm}
\textbf{Budapesti Műszaki és Gazdaságtudományi Egyetem}\\
\textmd{Villamosmérnöki és Informatikai Kar}\\
\textmd{Távközlési és Médiainformatikai Tanszék}\\[2.5cm]

\vspace{0.5cm}
\textsc{\LARGE Diplomaterv 1}\\[0.1cm]
\textsc{2015/16. tanév, 2. félév}\\[1.5cm]
\vspace{0.4cm}
{\huge \bfseries Automatizált internetes mérések}\\[5cm]

\begin{tabular}{cc}
 \makebox[7cm]{\emph{Készítette}} & \makebox[7cm]{\emph{Konzulens}} \\
 \makebox[7cm]{Horváth Rudolf} & \makebox[7cm]{Dr Heszberger Zalán}
\\
 \makebox[7cm]{rudolf.official@gmail.com} & \makebox[7cm]{heszberger@tmit.bme.hu}
 \\
 \makebox[7cm]{TS48JK} & \makebox[7cm]{}
\end{tabular}

\vfill
{\large \today}
\end{center}
\end{titlepage}




\tableofcontents

\include{nyilatkozat}

\chapter*{Kivonat}\addcontentsline{toc}{chapter}{Kivonat}
%1 oldalas összefoglalása a munkának \textbf{Magyarul}.

Az internet olyan kulcsfontosságú infrastruktúra napjainkban, amelynek fontos ismerni az állapotát és viselkedését. Ezen Diplomamunka az internetes útvonalak vizsgálatával foglalkozik.
Egy automatizált Internetes mérési rendszer készült ennek érdekében, amely a PlanetLab szervezet gépeit felhasználva méréseket végez és monitorozza az internet általuk látható részeit. A mérési rendszer több mérési funkcióval rendelkezik. A begyűjtött adatok több feldolgozási folyamaton mennek keresztül, melyek egyik kimenete például útvonal változásokat elemez. A megbízható adatszolgáltatás érdekében a mérési rendszer folyamatos és hibatűrő működésre lett tervezve, felügyeleti funkciókkal kiegészítve. A felhasznált technológiák és az elkészült szoftverkomponensek nyílt forráskódú licensszel rendelkeznek, így mások is használhatják azokat, illetve bővíthetik saját fejlesztéseikkel.
A begyűjtött adatok előzetes elemzése bemutatta, hogy a mérési rendszer valóban képes megfigyeléseket tenni az Internetes útvonalakra vonatkozóan. A publikált eredmények és eszközök értéket jelenthetnek minden internetes méréssel foglalkozó csoportnak.

\chapter*{Abstract}\addcontentsline{toc}{chapter}{Abstract}

Today, the Internet is a key infrastructure, it is important to know the behavior and current conditions of it. This Thesis deals with Internet route measurements, which helps to better understand, and tries to answer these question.
For this purpose, an automated measurement system was developed, which carries out internet measurements with the help of the PlanetLab network. The system is monitoring the parts of the internet seen by the nodes of the network. The collected measurements are stored and processed by multiple pipelines for different purposes, like route evolution observation. To provide a reliable data feed, the measurement system was designed for continuous and fault tolerant operation, extended with operational supervision functions. The used and developed technologies are accessible under Open-source licenses. Anybody has access to them and can further develop new components to it.
The collected data and its preliminary analysis demonstrated, that the measurement system can provide insights into the behavior of Internet routes. The publicized results and tools can have a real value for groups interested in internet measurements.


%Régi:
%Az internet olyan kulcsfontosságú infrastruktúra napjainkban, amelynek fontos ismerni az állapotát és viselkedését. Ezen Diplomamunka az internetes útvonalak vizsgálatával foglalkozik. Egy automatizált Internetes mérési rendszer készült ennek érdekében, amely a PlanetLab szervezet gépeit felhasználva méréseket végez és monitorozza az internet általuk látható részeit. A mérési rendszer több mérési funkcióval rendelkezik, a begyűjtött adatok több feldolgozási folyamaton mennek keresztül a jobb információprezentálás érdekében. Mindezeken felül a mérési rendszer folyamatos, megbízható és hibatűrő működésre lett tervezve.


%\chapter*{Abstract}\addcontentsline{toc}{chapter}{Abstract}
%1 oldalas összefoglalása a munkának \textbf{Angolul}.

%\chapter*{Bevezető}\addcontentsline{toc}{chapter}{Bevezető}
%A kivonatnál bővebb leírás miről szól a dolgozat.




%Önlab2 előszava:
%Az internet olyan kulcsfontosságú infrastruktúra napjainkban, amelynek fontos ismerni az állapotát és viselkedését. Önálló laboratóriumi feladatom során az internetes útvonalak automatizált mérésével és értelmezésével foglalkoztam. Az általam készített és továbbfejlesztett mérőrendszer a PlanetLab szervezet gépeit felhasználva méréseket végez és monitorozza az internet általuk látható részeit. Munkám során több új mérési funkciót fejlesztettem, valamint az új mérési forgatókönyvek implementálását egyszerűvé és kényelmessé tettem, a folyamatosan gyűlő adatokat könnyen feldolgozható formába alakítottam. Mindezeken felül fontos szempont volt a folyamatos, megbízható és hibatűrő működés, valamint a mérések felügyelete.



%%%%%%%%%%%%%%%%%%%%%%%%%%%%%%%%%%%%%%%%%%%%%%%%%%%%%%
%Elvégzett munka tartalma:

%Migráltam az adatmentés és adatkezelést MySQL-ről MongoDB-re, mert jobbnak tűnt, hogy nem merevek az adatstruktúrák és megférnek egymás mellett a régebbi formátumú adatok és a kísérleti, friss adatok is.
%A folyamatos futtatás érdekében a RedHat Openshift felhős platformjára telepítettem az egész mérést, így ott folyamatosan elindul, fut, ment egyből adatbázisba. Valamint egy honlapon mutat pár visszajelzést, hogyan áll: http://python-limiere.rhcloud.com/
%Több hibát javítottam, amiknek hála szinte bármilyen (akár új, váratlan) hiba előfordulásakor folytatják a mérést. Alapvetően függetlenítettem az egyes folyamatokat, amik felügyelni tudják a másikat és szükség esetén leállítják, újraindítják azokat.
%A hibák és a futások felügyeletéhez szinte mindenhol bevezettem a logolást, így végigkövethető hol milyen hiba miatt nem volt sikeres a mérés.
%Eddig csak traceroute mérés volt, de ezt kiegészítettem az iperf méréssel.
%Ennek nyomán pedig egy viszonylag szép általánosan bevethető eljárást készítettem, ahol különböző parancsok időzített, egymáshoz illesztett indítását lehet levezetni. Például ugye a párhuzamos méréshez kellett ez.
%Végül ugye elkészítettem az éllistából álló adatbázist amiben minden elérhető adat rendelkezésre áll (időpont, két ip cím, késleltetés, jitter ...)


%%%%%%%%%%%%%%%%%%%%%%%%%%%%%%%%%%%%%%%%%%%%%%%%%%%%%%
%Önlab2 eredeti kiírás:
%A labortéma a Forgalmi mérések az Interneten c. önálló labor kiírás testvértémája, és első sorban a mért adatok elemzését célozza: A laborfeladat célja méréseket végezni a nagy Interneten. A feladathoz az elmúl félévben korábbi önálló labormunka keretében készültek rövidebb python alapú szkriptek melyek az Interneten elérhető számos gép közötti forgalmat ill. annak viselkedését (pl. útvonalválasztási jellemzőit) mérik. Feladat lehet ezen mérések kivitelezése esetleg ha szükséges a scriptek továbbfejelsztése. A méréseket szeretnénk kielemezni, és következtetéseket levonni belőle a nagy Internet kinézetére és viselkedésére vonatkozólag.

%%%%%%%%%%%%%%%%%%%%%%%%%%%%%%%%%%%%%%%%%%%%%%%%%%%%%%
%Önlab1 előszava:
%Az internet fejlődése napjainkban nehezen követhető, hiába az alapjait alkotó protokollok alapos ismerete. A megvalósított hálózatok üzleti, jogi vagy természetes eredetű okok miatt sokszor eltérnek az eredetileg elképzelttől.
%Ezért az internet továbbfejlesztésének/újratervezésének egyik fontos feltétele a valóságos hálózatok ismerete. 
%
%Önálló laboratóriumi munkám során, hogy tapasztalatokat gyűjtsek az internet felépítéséről és a vizsgálatához szükséges technológiákról, méréseket végeztem a PlanetLab szervezet által elérhető számítógépek segítségével. A mérések célja az internet csomópontjai között lévő útvonalak és ezen útvonalak által létrehozott gráf feltérképezése és vizsgálata volt.


%------------------------------------------
%----------  Feladat kiírás  --------------
%\vspace{1.4cm}

%\noindent
%\textsc{\Huge \textbf{Feladat}}

%\bigskip
%%\indent

%%Önlab1 Feladat leírása
%A félév során a hallgató feladata a korábbi önálló laboratóriumi munkájának folytatása. A PlanetLab teszthálózat gépeinek segítségével végez internetes méréseket és tovább fejleszti azokat. A mérési eredményeket szűri és könnyen feldolgozható formában tárolja. Végül elemzéseket végez rajtuk.

%%%%%%%%%%%%%%%%%%%%%%%%%%%%%%%%%%%%%%%%%%%%%%%%%%%%%%
%Önlab1 Feladat leíása:
%A félév során a hallgató feladata az internet valóságos működésének és felépítésének vizsgálata. A témához tartozó irodalmakat tanulmányozza, majd a PlanetLab teszthálózaton automatizált méréseket folytat. Az így kapott mérési eredményeket feldolgozza és elemzi.




%%%%%%%%%%%%%%%%%%%%%%%%%%%%%%%%%%%%%%%%%%%%%%%%%%%%%%
%Önlab2 félév eleji feladat leírása:
%Féléves feladat:
%Internetes mérések tervezése, kivitelezése és elemzése,
%amely az internet felépítését elemzi
%és megvizsgálja az mptcp protokoll lehetséges előnyeit.
%
%A munka ütemezése:
%
%Előkészületek:
%4-5. hét  A kivitelezendő mérések kidolgozása. Teszt mérések végzése.
%
%Mérnöki munkavégzés:
%6-7. hét  Folyamatos mérések végzése és azok felügyelete.
%
%8-12.hét  A mérési eredmények aggregálása, feldolgozása és vizsgálata
%
%Dokumentálás:
%13.  hét  Felkészülés a szóbeli és írásbeli beszámolóra.




%%%%%%%%%%%%%%%%%%%%%%%%%%%%%%%%%%%%%%%%%%%%%%%%%%%%%
%Önlab1 eredeti kiírás:
%Az internet gyerekbetegségeinek tüneti kezelése mind komolyabb feladatot ró a szakemberekre. Egyre többen látják a megoldást az internet alapoktól való újragondolásában. Az általunk vizsgált kérdések érintik többek között a címzést, címkiosztást, útvonalválasztást, topológiát valamint az internet erősen elosztott és dinamikus jellegéből adodó nehézségeket. A laborfeladat célja megismerni az internet újragondolásának lehetséges új irányvonalait, ismertebb kezdeményezéseit. A félév második felében lehetőség nyílna alternatív hálózati technológiák elméleti és szimulációs és teszthálózati vizsgálatára ill. prototípus építésére.




%------------------------------------------------
%Elméleti áttekintés (16 oldal):

\chapter{Az Internet szerkezete és mérése}
%16 oldal

A következő fejezetekben az Internet struktúráját és működésének egyes aspektusait mutatom be, amelyek mélyebb megértést biztosítanak a mérési rendszer működésének és fontosságának megismerésében.
Az első szakaszban az internet makroszkopikus felépítését és működésének mechanizmusait mutatom be. Ezt követően az internetes útvonalak méréseire térek ki, amely során bemutatom a mérési rendszer által használt módszerek alapjait is. Végül korábbi és aktívan működő Internet mérési szervezetek munkáját és projektjeit mutatom be, amely segít felmérni a saját mérési rendszer értékes funkcióit és azok fontosságát.



\section{Az internet felépítése}
%(2-3 oldal): töri, AS, BGP, nemzetközi szervezetek

%első pár saját gondolat után wikipédia cuccok
%AS kialakulások
%tipikus AS-ek, tier-ek
%http://www.tmit.bme.hu/internet


Társadalmunk fejlődésében meghatározó szerepe van egyéb technológiai trendek mellett a szoros információs összeköttetésnek, amelyet az Internet biztosít. Történelmi tények említése nélkül az alábbiakban bemutatom a fejlődésének és működésének meghatározó részeit.

Az Internet előtt nagyobb szervezetek, egyetemek egymástól elszigetelt számítógépes hálózattal rendelkeztek. Idővel ezen szervezetek együttműködése révén a hálózataik között átjárást biztosítottak, új routing eljárások bevezetésével. Később ezen kezdetleges Internethez több és több további saját hálózattal rendelkező csoportok, Autonóm Rendszerek\footnote{Autonomus System - AS} csatlakoztak. Amikor egyre több internetes szolgáltatás született meg a polgári réteget is elkezdte érdekleni az Internetes elérés és a 1990-es években robbanásszerűen terjedt el az ISP-k segítségével lassan minden házastársba.

Általánosan elmondható hogy a Internetes hálózatban minden résztvevő maga felel a hálózatán belüli csomagtovábbításért, és üzleti-technikai együttműködések által érik el egymáson keresztül a globális Internetet. Az üzleti megállapodások biztosítják a pénzügyi hátterét a strukturális fejlesztéseknek és a minőségi, nagy távolságokat áthidaló összeköttetések létrehozásához, azok üzemeltetéséhez.

A következő szakaszokban bemutatom ezen résztvevőit/alkotóit az Internetnek. Kitérek többek között az együttműködésük részleteire, valamint a rájuk ható, őket szabályozó szervezetekre.

\subsection{Felügyelő szervezetek}
Az Internet egyik építőköve az IP\footnote{Internet Protokoll} adatcsomag, amelyben IP cím alapján van jelölve a forrás és a cél számítógép a hálózatban. Minden kommunikáció ezen alapszik, a csomagok továbbítása ez alapján történik. Ez a cím IPv4 esetén 32, IPv6 esetén 128 biten van ábrázolva. Az összes lehetséges cím kisebb címtartományokra van osztva, amelyeket a \ref{tab:cimtartomanyok} táblázat mutat be.

\begin{table}[ht]
	\centering
	\caption{Címtartomány kategóriák}
	\hspace{2mm}
	\begin{tabular}{ | c | c | c | r |}
	\hline
	Típus & Címtartomány & Tartalmazott címek száma & Ábrázolási példa \\ \hline

Class A & 24 bit & 16777216 db & 10.0.0.0/8\\ \hline
Class B & 20 bit & 1048576 db & 172.16.0.0/12\\ \hline
Class C & 16 bit & 65536 db & 192.168.0.0/16\\ \hline
	\end{tabular}
	\label{tab:cimtartomanyok}
\end{table}


Ezen tartományokat független nemzetközi szervezetek osztják szét az egész világon. Legfelsőbb szinten az IANA\footnote{Internet Assigned Numbers Authority - Szabad fordításban: Internetes számok felügyete} rendelkezik az egyes IP címtartományokkal. A kiosztásokat bárki szabadon megtekintheti a szervezet honlapján\footnote{\url{http://www.iana.org/assignments/ipv4-address-space/ipv4-address-space.xhtml}}.

Az IANA alá tartoznak tözvetlenül a következő Regionális Internet Nyilvántartó szervezetek\footnote{RIR - Regional Internet Registries}: APNIC, ARIN, RIPE NCC, LACNIC, AFRINIC.
Ezen szervezetek területi felelősségeit mutatja a \ref{fig:rir-regions} ábra.

\begin{figure}[!ht]
	\centering
	\includegraphics[width=0.9\textwidth, keepaspectratio]{figures/rir-regions.png}
	\caption{Regionális Internet Nyilvántartó szervezetek és felügyelt területeik \protect\footnotemark}
	\label{fig:rir-regions}
\end{figure}

\footnotetext{forrás: \url{https://www.apnic.net/about-APNIC/organization/history-of-apnic/history-of-the-regional-internet-registries}}

\newpage

Az egyes végfelhasználók ezen szervezeteken keresztül közvetlenül vagy közvetítéssel juthatnak saját IP címhez. Amely folyamatot részletesebben a \ref{fig:ipv4_hierarchy} ábra mutat be, amelyen már a közvetítő szervezetek is be vannak mutatva.

\begin{figure}[!ht]
	\centering
	\includegraphics[width=0.7\textwidth, keepaspectratio]{figures/ipv4_hierarchy.png}
	\caption{IP tartományok kiosztásának hierarchiája\protect\footnotemark}
	\label{fig:ipv4_hierarchy}
\end{figure}

\footnotetext{forrás: \url{https://www.apnic.net/policy/ipv6-address-policy_obsolete}}


\subsection{Autonóm rendszerek}

Az egyes címtartományokkal rendelkező szervezetek, Autonóm Rendszereket alkotnak, amelyek saját hálózattal rendelkeznek. Ezen hálózatukon teljes hatalommal és felelősséggel rendelkeznek. A többi autonóm rendszerrel akkor lépnek kapcsolatba, amikor a címtartományukon kívüli célnak kell elérést biztosítani vagy más autonóm rendszer szeretne elérést kapni egy az IP tartományába eső célgéphez.
Ezen forgatókönyvek esetén már nem a saját belső útvonal választási stratégiájukat használják, hanem az Autonóm Rendszerek között szabványosítottan működő protokollok alapján járnak el. Ez a BGP-nek (Border Gateway Protocol) hálózati protokoll, amelyet minden Autonóm Rendszer külső csatlakozási pontján lévő útvonalválasztó használ. 

Ezen speciális útvonalválasztók szigorú üzleti szerződések alapján működnek, mivel az átmenő forgalom komoly pénzügyi következményekkel jár legtöbb esetben. 

%content/eyeball/tranzit AS
%access/edge/core
%tier 3, 2, 1
%provider, customer
%transit, peering, IXP internet exchange point,
%stub AS: single homed, multihomed, 









%Internet felépítése (4-5 oldal): töri, AS, BGP, nemzetközi szervezetek


\section{Internetes mérési eljárások}
(5-6 oldal): Aktív/Passzív, Looking Glass, traceroute, ping, iperf
whois szerverek
 
 
hivatkozás a cikkre: http://www.ijcsit.com/docs/Volume%202/vol2issue4/ijcsit2011020402.pdf

ICMP TTL


%Internetes mérések (5-6 oldal): Aktív/Passzív, Looking Glass, traceroute, ping, iperf


\section{Internet felderítő projektek} \label{internet-measurement}

%(6 oldal): CAIDA, Archipelago és egyéb mérőrendszerek, PlanetLab, útvonal eltérítések esetei


Jelen fejezetben a legfontosabb Internet felderítését célzó szervezeteket és projekteket ismertetem, illetve megemlítem a legfontosabb ezeket felhasználó cikkeket.


\subsection{Internet elemző központ: CAIDA}

%\textbf{2-3 oldal kb}

A CAIDA (Center for Applied Internet Data Analysis)\footnote{\url{http://www.caida.org/home/}} egy kollaborációs együttműködése állami, kutatói és kereskedelmi szervezeteknek aminek célja az Internet üzemeltetésének, felügyeletének és fejlesztésének segítése. Alapvetően az alábbi három fő területre koncentrálódnak a szervezet munkakörei és projektjei:

\begin{itemize}
\item Kutatás és analizálás
%Research and Analysis	Analyze
\item Mérés és infrastruktúra
%Measurements and Infrastructure
\item Adat és eszközök.
%Data and Tools.
\end{itemize}

Munkájuk során az Internet egyes aspektusainak megismeréséhez modelleket készítenek, majd vizsgálják azok érvényességét, végül publikálják a talált összefüggéseket. Céljuk előremozdítani az Internet növekedésének problémamentességét. Ennek érdekében folyamatosan feltérképezik az Internet szerkezetét, amelyet a korábbi fejezetekben is bemutattam. Évente új \mbox{\glqq térképet \grqq} készítenek az Autonóm Rendszerek hálózatáról, mind az IPv4, mind az IPv6-os hálózatrészekről külön-külön. A \ref{fig:caida-poster} ábrán látható gráf pontjainak (AS) elhelyezése a földön is betöltött szélességi foka alapján (irány) és a rajta átmenő linkek száma alapján vannak elhelyezve (sugár, logaritmikusan ábrázolva).

\begin{figure}[!ht]
	\centering
	\includegraphics[width=0.9\textwidth, keepaspectratio]{figures/caida-poster.png}
	\caption{Az Internet AS gráfja 2014-ben}
	\label{fig:caida-poster}
\end{figure}

Az általános megfigyeléseken felül kutatásaik a kibervédelemre is kiterjednek. Céljuk az Internet helyes működése elleni tevékenységek felderítése és utólagos megértésük.


Ezen széles skálájú megfigyelésekhez kiterjedt mérési rendszereket üzemeltetnek a világ minden táján. Többek között begyűjtik az összes publikusan elérhető BGP útvonalhirdetéseket és a kooperáló szervezeteknél elhelyezkedő traceroute szerverekkel felméréseket végeznek a csomagtovábbítási viszonyokról az Internet egyes részeiben. Mindezeken felül pedig több Autonóm Rendszer hálózatában passzív méréseket is végeznek. Ennek a felhasználására egy példa az internetes háttérzaj mérése, ami az olyan adatcsomag küldések intenzitását mérik, amelyek olyan cél címekkel rendelkeznek ahol nem létezik fogadó számítógép. Az ilyen csomagok általában botnetek terjedési próbálkozásának a terméke.

Az említett adatforrások egy központi gyűjtő és organizáló rendszerbe, az ARK\footnote{Archipelago}-ba vannak csatornázva. A feldolgozás során rendkívül nagy hangsúly van fektetve a kinyert adatok hivatalosításán. Minden származtatott adatban meg vannak jelölve annak forrásai, valamint a különböző forrásokból származó azonos információk gondosan össze vannak vetve, ellenőrizve megbízhatóságukat.

%további tartalom majd innen: http://www.caida.org/home/about/progplan/progplan2014/
%% fél oldal kb eddig

%második oldal a 3 fő tématerületének a kifejtése, témánként fél oldal.

%Számunkra legfontosabb.. felsorolás a projektjeik közül

%%%%%%%%%%%%%%%%%%%%%%%%%%%%%%%%%%%%%%%%%%%%%%

\subsection{PlanetLab}

A PlanetLab konzorcium egyetemi, ipari és kormányzati intézmények együttműködő csoportja, amelyek erőforrásaikat megosztják kutatási céllal, így építik és fejlesztik a PlanetLab számítógépes hálózatát. A résztvevő szervezetek így hozzáférnek a sajátjuknál nagyobb hálózathoz. Ilyen módon a résztvevők kivételes lehetőséghez jutnak, hogy  nagyméretű hálózatokon és elosztott rendszereken tudjanak vizsgálatokat végezni.

A konzorciumot 2002-ben alapította 10 amerikai egyetem az Intel Research közreműködésével, kezdetben 100 számítógép-csomópont létrehozásával. A közösségi teszthálózat évről évre bővült újabb egyetemek és ipari kapcsolatok csatlakozásával. A PlanetLab 2004-ben több Amerikán kívüli szervezetet is befogadott többek között európából, brazíliából és ázsiából. Ebben az évben így már világméretűvé nőtt és több mint 400 számítógép csomóponttal rendelkezett. Napjainkban ez a szám már túllépte az ezret és az \ref{fig:planetlab} ábrán látható, hogy a föld minden táján megtalálhatóak a hálózatban résztvevő csomópontok.

A fejlesztett mérőrendszer is a PlanetLab hálózatát felhasználva végez méréseket az internet szerkezetére vonatkozóan.

\begin{figure}[!ht]
	\centering
	\includegraphics[width=0.65\textwidth, keepaspectratio]{figures/planetlab_worldmap.png}
	\caption{A PlanetLab teszthálózat csomópontjainak elhelyezkedése a világtérképen\label{fig:planetlab}}
\end{figure}

\subsection{M-Lab}
%2 oldal

Az előzőekben bemutatott PlanetLab és további piaci és kutatói szervezetek, többek között a Google, által alakított csoport, melynek célja az Internetes kapcsolatok végfelhasználói szemszögből való elemzése. Egyik legfontosabb célja a felhasználói élmény szűk keresztmetszetének a felderítése.

Begyűjtött adataik fontos része ilyen módon az önkéntes mérésekből származnak. Méréseket a weboldalukról, mobileszközről és telepített szoftverekkel lehet indítani. Minden adatuk nyilvánosan elérhető és letölthető. Több kutatási projektet is támogatnak mérési adataikkal. Naponta több mint 200 ezer tesztet futtatnak, melyek 750 TB adatot gyűjtöttek össze 2015-ig. 


%Elég nagy és működő félig üzleti szervezetnek tűnik. A Google is részese, gondolom ellenőrizni akarják hogy az ISP-k jól működnek-e. Egyik eszközük például külön kutatja, hogy mi a legszűkebb keresztmetszet az ügyfél kapcsolatában.
%Honlapjuk: http://www.measurementlab.net/
%Mobil mérésekkel is foglalkoznak, készítettek egy mobil applikációt ami adatokat gyűjt mérésekről. 2012 és 2013-ban kapott adatok elérhetőek: https://www.measurementlab.net/tools/mobiperf/

\subsection{SamKnows}
%1 oldal

Az Európai Unió és az amerikai FCC is felkarolta az egyetemi projektből kifejlődő SamKnows\footnote{\url{https://www.samknows.com/}} cég működését. Eredetileg a hétköznapi emberek könnyebb és objektív ISP választását segítette a projekt. Kezdetben önkéntesek végeztek méréseket a fejlesztett rendszerükön, hogy megállapítsák a szolgáltatójuk kapcsolatának minőségét. Nem csak sávszélesség mérésre alkalmas, de további hálózati paraméterek mellett a szolgáltatások diszkriminációját is figyeli. Olyan speciális adatcsomagokat használ a mérésekhez, amelyek HTTP, FTP, torrent, youtube, netflix, vagy VOIP\footnote{Voice Over IP - IP alapú hangkapcsolati protokoll} forgalmat szimulálnak. Ezen funkciói miatt rendkívül hasznos ellenőrzési eszköze lett a felügyeleti hatóságoknak. Folyamatosan fejlesztik a rendszerüket és riportokat készítenek a szolgáltatókról és működésükről.
%Az korábbi Internetsemlegességi vitáknál is felhasználták az így gyűjtött adatokat.
Mérési rendszerük támogatja a weboldalon keresztüli méréseket, mobil applikáción keresztüli méréseket és különböző telepíthető hardverelemeket is felhasznál, mint például a SamKnows Whitebox.

\begin{figure}[!ht]
	\centering
	\includegraphics[width=0.45\textwidth, keepaspectratio]{figures/samknows-whitebox.png}
	\hspace{20pt}
	\includegraphics[width=0.45\textwidth, keepaspectratio]{figures/samknows-whitebox2.png}
		
	\caption{A SamKnows Whitebox méréseinek vázlata és elhelyezkedése a háztartásokban}
	\label{fig:samknows}
\end{figure}

A Whitebox készülékek működési elvét a \ref{fig:samknows} ábrák mutatják be, amely a korábban említett méréseket végzi el egy a háztartás hálózatába kötött eszköz segítségével. Az Európia Unió 2012-ben kérte fel a SamKnows céget a tagállamok interneteléréseinek a vizsgálatára. A tanulmány\footnote{\href{https://ec.europa.eu/digital-single-market/news/quality-broadband-services-eu-samknows-study-internet-speeds}{Quality of Broadband Services in the EU}} során 9467 háztartásba helyezték el a Whitebox készülékeket. A mérési adatok publikusan elérhetőek a \url{data.europa.eu} honlapról


\subsection{A DIMES mérési rendszer}
%1.5-2 oldal

2010 és 2013 között több egyetem bevonásával egy Izraeli kutatócsoport indította el a DIMES (Distributed Internet Measurement and Simulation) kutatóprojektet. Céljuk az Internet struktúrájának és topológiájának a megismerése volt, amelyet önkéntes módon segítenek további résztevők.
Az önkéntesek szerepe a geológiai diverzitás volt, amelynek segítségével több megfigyelési pontból lehet megvizsgálni az Internet szerkezetét.
Az elvégzett mérések egyszerű Ping és Traceroute mérésekből álltak, amelyek minimális erőforrásigényt támasztottak az önkéntesek gépein. A résztvevők cserébe személyre szabott jelentéseket kaptak az internetes kapcsolatuk minőségéről.

Kutatási eredményeik felhívták a figyelmet az Autonóm Rendszerek rendkívüli változatosságára és kapcsolataik nehezen felderíthetőségére. Egyik fontos megállapításuk a CAIDA akkori felméréseihez képest, hogy a Looking Glass szervereken keresztül elérhető BGP hirdetésekből hiányoznak az egyes AS-ek peering egyezményeikből fakadó összeköttetések.

A kutatási projekt aktivitása 2013 óta teljesen megszűnt, a hivatalos honlapjuk nincs karbantartva.

%Honlapja: http://www.netdimes.org/new/
%A DIMES honlapja gyakorlatilag nem működik, mindenféle hibákba ütközik (de legalább elérhető). A legutolsó hír rajta vagy bejegyzése 2013-ban íródott. Nem erléhetőek sajnos így az általuk elért adat sem, sem az infók róla :(
%A publikációjukból tudunk legjobban kiindulni: link (2000 és 2013 közöttiek)
%A saját mérési adataik elérhetőek: link
%Lehet érdemes felvenni velük a kontaktot, hogy külsős fél mennyire tudná most futtatni?
%Email: support@netdimes.org
%A projekt résztvevői: link
%A projekt fóruma: link
%A fórumon van egy “friss” szál ahol megbeszélik, hogy ez egy halott projekt (link).
%Ebben a beszélgetésben egy lényegretörő hozzászólás:
%Since I found the project I've now gotten rid of all Windows machines, and the Linux version never worked anyway. I can't run the agent either way.

\subsection{Kisebb Internet mérő projektek}

A fentiekben felsoroltakon kívül még számos kisebb internet mérést célzó projekt létezik még melyek közül a továbbiakban kettőt emelek ki.

\subsubsection*{Az IRL mérési rendszer}
 
 
Egyetemi projektként\footnote{\url{http://irl.cs.tamu.edu/projects/sampling/}} kutatásuk középpontjában a teljes Internetre vonatkozó mérések hatékonyságának a növelése állt. Új eljárásokat fejlesztettek ki, melyekkel nem a bevett elárasztásos módon derítik fel az egyes hálózati tartományokat. Utóbbi publikációjukban\cite{irl-measure} pedig a hálózatban résztvevő eszközök operációs rendszerét állapították meg.
 

%Sampling Internet structure and its service availability have always been important issues in Internet research. This project aims to develop mechanisms for discovering available services in the Internet using scalable end-to-end measurements, facilitate delay sampling between arbitrary hosts using the existing DNS infrastructure, perform more accurate bandwidth estimation, and build router-level maps of the Internet using non-intrusive traceroute to known destinations. We are also working on techniques for fingerprinting OS and server implementations using low-overhead end-to-end methods.


\subsubsection*{Az iPlane mérési rendszer}
 


Diploma munkaként született, majd kutatási projektté fejlődött az iPlane\footnote{\url{http://iplane.cs.washington.edu/}}, melynek fő célja az internetes útvonalak QoS becslése és akár predikciója. Kimutatásaikat a PlanetLab gépeken és további publikus Traceroute szervereken futtatott útvonal felderítő mérések alapján végezték. Céljuk a felhalmozott adatok felhasználása későbbi szolgáltatások minőségének javítására.Elemzéseik fontos része, hogy a felderített útvonalaknak a rész szakaszaira becsülnek paramétereket, így még nem mért útvonalak viselkedésére is tudnak becslést szolgáltatni. Ezt a becslést pedig Peer-to-peer és Voice-Over-IP alkalmazások működésének javítására használták.

%Alapvetően egy mellék képessége az internetes utak  emgfigyelése, mérése. A lényege az internetes útvonalak QoS becslése, jóslása a korábbi mérésekből, valamint ennek a felahsználása a felsőbb szolgáltatások javítására. Szintén PlanetLab os gépeket használ, valamint további publikus traceroute szervereket a mérések elvégzésére.
%Rengeteg traceroute mérést végeznek, még napjainkban is, ezek elvileg elérhetőek bárki által. (Akár mi is hasznosíthatnánk?)
%Nem elérhető a rendszerük forrása, kicsit homályos a működése (valószínűleg a publikációjukból meg lehetne ismerni).
%Az iPlane egy Diploma  munka eredménye, itt van maga a dimploma: link

%iPlane is a scalable service providing accurate predictions of Internet path performance for emerging overlay services. Unlike the more common black box latency prediction techniques in use today, iPlane builds a structural model of the Internet. We construct an annotated map of the Internet and predict end-to-end performance by composing measured performance of segments of known Internet paths. This method allows us to accurately and efficiently predict latency, bandwidth, capacity and loss rates between arbitrary Internet hosts. We have studied the feasibility and utility of the iPlane service by applying it to several representative overlay services in use today: content distribution, swarming peer-to-peer filesharing, and voice-over-IP. In each case, we observe that using iPlane's predictions leads to a significant improvement in end user performance. 
%Internet felderítése (6 oldal): CAIDA, Archipelago és egyéb mérőrendszerek, PlanetLab

%------------------------------------------------
%Mérőrendszer felépítése (13-15 oldal):

\chapter{Automatizált Internet mérő rendszer}
(13-15 oldal): 


\section{Mérési elrendezés}
(5 oldal): traceroute, iperf, mérési forgatókönyvek, PlanetLab

\subsection{A PlanetLab hálózata}

A PlanetLab konzorcium egyetemi, ipari és kormányzati intézmények együttműködő csoportja, amelyek együttműködése fejleszti és támogatja a PlanetLab számítógépes hálózatot. A résztvevő szervezetek megosztják saját erőforrásaikat és hozzáférnek a hálózatba kötött számítógépekhez. Ilyen módon a résztvevők kivételes lehetőséghez jutnak, hogy  nagyméretű hálózatokon és elosztott rendszereken tudjanak vizsgálatokat végezni.

A konzorciumot 2002-ben alapította 10 amerikai egyetem az Intel Research közreműködésével, kezdetben 100 számítógép-csomópont létrehozásával. A közösségi teszthálózat évről évre bővült újabb egyetemek és ipari kapcsolatok csatlakozásával. A PlanetLab 2004-ben több Amerikán kívüli szervezetet is befogadott többek között európából, brazíliából és ázsiából. Ebben az évben így már világméretűvé nőtt és több mint 400 számítógép csomóponttal rendelkezett. Napjainkban ez a szám már túllépte az ezret és az \ref{fig:planetlab} ábrán látható, hogy a föld minden táján megtalálhatóak a hálózatban résztvevő csomópontok.

\begin{figure}[!ht]
	\centering
	\includegraphics[width=0.65\textwidth, keepaspectratio]{figures/planetlab_worldmap.png}
	\caption{A PlanetLab teszthálózat csomópontjainak elhelyezkedése a világtérképen\label{fig:planetlab}}
\end{figure}

\subsection{Csatlakozás a hálózat számítógépeihez}

A tervezett mérések elvégzéséhez szükséges volt a csomópontokhoz való automatizált csatlakozás és parancs végrehajtás. Ehhez a PlanetLab központi szervere szolgált egyszerűen lekérhető listát a hálózatban résztvevő csomópontok címéről. A mérések elvégzéséhez és feldolgozásához egységesen Python programozási nyelven megírt szkripteket használtam. A méréseket végző 
program a Paramiko\cite{paramiko} könyvtárat használja az ssh kapcsolatok felépítéséhez és menedzseléséhez. A mérési kísérletek sok esetben hiúsultak meg különböző hibák miatt, vagy a célszámítógép elérhetetlensége miatt, ezért ezen esetek kezelése fontos szempont volt a fejlesztés során. Az \ref{fig:statistics} ábrákon látható statisztikák a márciusi mérések adataiból készültek. A mérések óránként lettek elvégezve, lefutásonként 1354 csomóponthoz téve csatlakozási kísérletet. A vezérlő számítógép nem állandóan futott, így a hónapban csak 23 nap készültek mérések, átlagosan 15 alkalommal.

\subsection{Időzített mérési forgatókönyvek}
Az iperf mérések implementálásakor több új eljárás kidolgozására volt szükség a mérést menedzselő szoftverben. A korábbi traceroute mérések egyetlen parancs távoli futtatásából álltak, amelyek futási eredményét tároltuk el mérési eredményként. Az iperf és más sávszélesség mérő szoftverek működéséhez azonban szükséges mind egy adatcsomagokat küldő és egy adatcsomagokat fogadó példány futtatása a két különböző távoli gépen. Ennek lebonyolításához a mérést lebonyolító programnak több szálon kell futnia, a két távoli kódfuttatást egyszerre kell végeznie. Először az adatcsomagok fogadását (szerver oldal) végző programot kell elindítani, majd csak ezt követően lehet csak a mérést megkezdeni az adatcsomagokat küldő program (kliens oldal) elindításával. A mérést követően a szervert pedig le kell állítani. Ez a szituáció tovább bonyolódik, ha két sávszélesség mérést párhuzamosan szeretnénk végezni. Ez egy fennálló igény, mivel a későbbiekben az MPTCP\footnote{Multipath Transfer Protocol: Több párhuzamos TCP adatfolyamon végez kommunikációt a felsőbb rétegek felé egyetlen TCP kapcsolatot emulálva.} protokoll lehetséges viselkedését is vizsgálni szeretnénk.

Ezeket a lépéseket általánosítva olyan mérési forgatókönyvek létrehozását támogatja már a mérési rendszer, amely bármilyen távoli parancsok időzített futtatását garantálja. Ennek kialakítása a lehető legrugalmasabbra lett tervezve, amelynek működését a függelékben található példakód mutatja be. A kód egyszerűségének ellenére a mérés teljesen menedzselt, bármilyen hiba keletkezése le van kezelve és megfelelően naplózva és a mérési eredményben jelezve van. Garantálva van a helyes időzítés, a párhuzamos futás és a helyes leállás.

A mérési forgatókönyvek rendkívül hasznos eszközzé váltak, a segítségükkel új mérések implementálása kényelmes és gyors.
%Mérési elrendezés (5 oldal): traceroute, iperf, mérési forgatókönyvek, PlanetLab


\section{Fejlesztési megfontolások}
%(8-10 oldal): python, mongodb, logolás, email küldés, weboldal, hibák függetlenítése, megbízhatóságnövelő megfontolások


\subsection{MongoDB adatbázis használata}
Először SQLite adatbázis volt használva a mérési eredmények tárolására. Az SQL lekérdezési nyelvet széleskörű ismerete miatt lett használva, valamint a Python programozási környezetből való kényelmes használta miatt. Az adatbázis egyszerű fájlként van tárolva, így könnyebben kezelhető. A fejlesztés közben sűrűn változó struktúrájú adatok tárolásár azonban nem volt megfelelő. Egy új mérési attribútum bevezetésekor új adattáblát kellett létrehozni, a régi adatokat pedig megfelelően importálni. A felmerülő adatkonverziók feleslegesen vettek el időt. Ezen felül az adatok feldolgozása is erőforrás igényes volt, mivel a komplexebb adatfeldolgozási lépések Python szkripteken futottak amelyek többször fordultak újra az SQLite adatbázishoz.

Ezen problémák miatt más adatbázis megoldások lettek felmérve, amelyek jobban illeszkedhetnek majd a rendszerek igényeihez. A relációs adatbázisokkal szemben a dokumentum alapú adatbázisok nem rendelkeznek szigorú adatstruktúrákkal. Egy kollekcióban (az adattáblákhoz hasonló elem) többféle felépítésű adatelem is elhelyezkedhet. Egy hibás mérés adateleme például csak kevéssel különbözik a többitől. Ezek könnyen kezelhetően együtt tudnak élni egy ilyen dokumentum alapú adatbázisban, míg komplex relációs táblák kialakítása lett volna szükséges SQL esetében.

Végül a MongoDB lett kiválasztva, a széleskörű adatfeldolgozási támogatása miatt. Egyes komplexebb feldolgozási lépéseket akár azonnal az adatbázisban el lehet végezni, a többfokozatú adataggregálási rendszerének és a JavaScript alapú map-reduce eljárásnak hála. Az adatfeldolgozó képességeinek bemutatására az alábbi kódrészletet mutatom be, amely egy későbbiekben bemutatandó adat kollekcióból kigyűjti a mérések által felderített Internetes kapcsolatok tulajdonságait:

\begin{lstlisting}[language=JavaScript]
db.getCollection('links').aggregate([
    {$project:{_id:0, links:1}},
    {$unwind:"$links"},
    {$project:{
        delay: "$links.delay",
        jitter: "$links.jitter",
        rtt: "$links.rtt"
    }},
    {$out:"to_export"}
])
\end{lstlisting}

A választás meghozta eredményét, mivel felgyorsult az új adatfeldolgozások fejlesztése, valamint a régebbi struktúrájú mérések az újakkal könnyen együtt kezelhetővé váltak.

%%%%%%%%%%%%%%%%%%%%%%%%%%%%%%%%%%%%%%%%%%%%%%%%%
% Címszavakban a tartalom:
%rugalmas adatstruktúrák
%fejlesztés közbeni átmeneti állapotok jó kezelése
%Aggregálási képességek hangsúlyozása
%(komplex lekérdezése)


\subsection{Megbízhatósági törekvések}
A mérési rendszer úgy lett kialakítva, hogy bizonytalan körülmények között is megfelelően működjön. A PlanetLab hálózatában rendkívül sokféle számítógép található, amelyek karbantartása sok esetben nem megfelelő. Korábban olyanra is akadt példa, hogy egy PlanetLab-os számítógép rendellenes viselkedése leállította az egész mérési folyamatot. Az ehhez hasonló hibák elkerülésére valamint a hibák minél hamarabbi észlelésére nagy figyelem lett fordítva a mérési rendszer fejlesztésekor.

\subsubsection*{Hibalehetőségek függetlenítése}
A mérés egy folyamatos ciklusból áll: A PlanetLab gépeit megpróbálja elérni, azokon a megfelelő méréseket elvégezni, végül eltárolni az eredményeket, majd egy újabb géppel folytatni a mérést.
Korábban egy hiba fenntarthatta az egész mérést. Ennek elkerülése miatt két programszálra lett bontva a folyamat: Egy központi program folyamatosan iterál a PlanetLab gépein és egy újabb független programot indít a mérendő gép címét paraméterként megadva. A hívott program bármilyen hibába ütközik arról értesül a főprogram, de nem tudja leállásra kényszeríteni azt. Az alprogram futásának pedig egy félperces maximálisan megengedett időablaka van, amelyet ha meghalad automatikusan leállításra kerül és egy újabb méréssel folytatódik a ciklus.

\subsubsection*{Maximálisan megengedett időkeret elve}
Az maximálisan megengedett időkeret elvét több másik folyamatra is bevezettem. A mérés során ha bármely lépése a folyamatnak időkorlátba ütközik a hiba oka fel lesz tüntetve a mérésről készült jelentésben. Ilyen módon könnyen kideríthető a hiba oka.

\subsection*{Hibanaplók készítése}
A fejlesztés kezdeti szakaszában ha hiba fordult elő a kódban az viszonylag hamar kiderült és a fejlesztőkörnyezetben kényelmes eszközök álltak rendelkezésre, hogy a kiderüljön a hiba oka. Később amikor a mérés folyamatosan futott nagyban feltartotta a megoldását, hogy a hiba egyetlen jelensége az volt, hogy nem került mentésre egy újabb sikeres (vagy akár sikertelen) mérés sem. Ennek megoldására a mérésnek szinte miden szintjén be lett vezetve a hibanapló írása. Így a mérés állapota és az esetleges hibák forrásai egyből elérhetővé váltak. A felhős platformon pedig a hibanaplók akár a weboldalon egyből megtekinthetővé váltak.

\subsubsection*{Felmerülő hibák kategorizálása, előfordulásuk monitorozása}
Mint korábban említettem a PlanetLab hálózatán sok nem karbantartott gép is jelen van, emiatt a hivatalosan elérhető 1030 gépből átlagosan csak 400 elérhető ping paranccsal, amelyek közül átlagosan csak 200-hoz lehet sikeresen ssh kapcsolatot felépíteni és parancsot futtatni. A géppark állapotáról ezért óránként egy felmérés készül, amely a PlanetLab gépeihez csatlakozni próbál és a problémamentes gépeket nyilvántartja, hogy ne kelljen feleslegesen csatlakozási kísérletet végezni a hibás gépekhez. Ez az eljárás nagyban felgyorsította a sikeres mérések begyűjtését.

A csatlakozási és parancsfuttatási hibák a felmérés során kategorizálásra kerülnek és későbbi statisztikák számára tárolódnak. Így nyomon követhető a PlanetLab hálózatán elérhető gépek tényleges elérhetősége/használhatósága.
A mérési rendszer honlapján ezek a friss statisztikák szintén elérhetően, így könnyen felmérhetjük a tömeges hibák okait. Ilyenre volt példa amikor a BME által üzemeltetett PlanetLab gépek leálltak és a PlanetLab megszüntette a hálózatukban elérhető gépekre való csatlakozási engedélyünket. Egy új hibakategória jelent meg a felmérési statisztikákban amely hirtelen sok gépet tett elérhetetlenné (bár nem mindet).

%Fejlesztési megfontolások (8-10 oldal): python, mongodb, logolás, email küldés, weboldal, hibák függetlenítése, megbízhatóságnövelő megfontolások

%------------------------------------------------
%Mérési eredmények áttekintése (12-15 oldal):

\chapter{Mérési eredmények}
%%%%%%%%%%%%%%%%%%%%%%%%%%%%%%%%%%%%%%%%%%%%%%%%%
%(12-15 oldal)



%\section{Mérési eredmények}
%(7-10 oldal): kinyert mérések tartalma(delay, jitter, link, AS info, Geolocation), mérések mennyisége, mérések minősége

\section{Létrehozott adatbázis bemutatása}
Jelen fejezet bemutatja a létrehozott adatbázis fontosabb adathalmazait, amelyek további elemzések alapját képezi.

\subsection*{Éllista}
A legfontosabb eredmény az internetes útvonalakat alkotó gépek közötti kapcsolatokról szolgáltat mérési eredményeket. Az objektum amelyről a mérés készült egy internetes link, a kinyert információk a következők:

\begin{itemize}
\item \textbf{delay:} Késleltetés a két gép közötti kapcsolaton (két gép közötti elméleti rtt)
\item \textbf{rtt:} A from számítógéphez végzett körülfordulási idő a mérést végző measurer\_ip számítógéptől
\item \textbf{time:} A mérés időpontja
\item \textbf{jitter:} Késleltetés ingadozás a két számítógép között
\item \textbf{measurer\_ip:} A mérést végző számítógép (ahol a traceroute parancs fut)
\item \textbf{target\_ip:} Az útvonalmérés célpontja
\item \textbf{to:} A linkek a mérést végző számítógéptől távolabbi csomópontjának információi: city, country, longitude, latitude, ip, asn
\item \textbf{from:} A linknek a mérést végző számítógéphez közelebbi csomópontjának információi: city, country, longitude, latitude, ip, asn
\end{itemize}



Az elkészült mérési rendszernek ez az adat kollekció\footnote{A MongoDB adatbázisban táblák helyett kollekciókban tárolódnak az adatbejegyzések} az egyik legfontosabb eredménye. A Traceroute parancs futtatásának kimenetéből lett feldolgozva\cite{traceParse} és további adatmezőkkel felgazdagítva.

\begin{figure}[h!]
	\centering
	\includegraphics[width=0.95\textwidth, keepaspectratio]{figures/route_rtt_hist_max.png}
	\caption{Két végpont közötti késleltetés eloszlása}
	\label{fig:rtt-hist}
\end{figure}

A \ref{fig:rtt-hist} ábrán az egyik legalapvetőbb kimutatás látható, a késleltetések eloszlását a PlanetLab gépe között. Megfigyelhető az eloszlás lecsengése 350 milliszekundum felett, azonban a kimutatásból le lett vágva az rtt értékek felső 0.5\%-a. Ezek a mérések az esetek legnagyobb részében ideiglenes hálózati problémák alatt készültek, amelyek nem reprezentatívak a linkek tényleges tulajdonságaira vonatkozóan.

Az eloszlás első csúcsának elkülönülése a többitől az Internetet alkotó gráf útvonalainak a valós világhon alapuló tulajdonsága határozza meg. A mérési csomópontok túlnyomó része Európán vagy Amerikán belüliek, ezt magyarázza a 42 milliszekundum körüli első csúcs. Amint viszont egy kontinenseket átívelő linket érint a mérés, legalább 70 milliszekundumos késleltetés hozzáadódik az addigiakhoz, így adva a 113 milliszekundum körüli újabb csúcsot. A 70 milliszekundumos késleltetést a London-New York\footnote{Forrás: \href{https://wondernetwork.com/pings/New+York/London}{wondernetwork.com}} közötti tipikus késleltetés alapján lett figyelembe véve. A további késleltetés csúcsok további kontinensek közötti késleltetések alapján származtathatóak. Az hasonló kimutatások jól prezentálják az Internetnek a valós világunkhoz való szoros kapcsolatát.

Az rtt-ből származtatott delay adat az Internetet alkotó linkek és azok megfigyelhetőségére nyújt betekintést. Az útvonalakon lévő hálózati eszközök eltérő módon válaszolnak a lejárt TTL mezőjű adatcsomagokra, amelyeket a traceroute használ. Ennek legegyértelműbb bizonyítéka a delay adatsor kvantiliseinek értékei:

\renewcommand{\arraystretch}{1.3}
\begin{table}[ht]
	\centering
	\caption{A delay adatsor kvantilisei}
	\hspace{2mm}
	\begin{tabular}{ | c | c | c | c | c |}
	\hline
0\% & 25\% & 50\% & 75\% & 100\% \\ \hline
-2217.682 & -0.188 & 0.631 & 6.775 & 2323.324 \\ 
\hline
	\end{tabular}
	\label{tab:delay_kvant}
\end{table}

A \ref{tab:delay_kvant} táblázatból az olvasható le, hogy a delay értékeknek legalább 25\%-a negatív értékű és széles tartományon -2 szekundumtól +2 szekundumig terjed. A meglepően nagy arányban előforduló negatív értékek magyarázata az, hogy míg egy hálózati eszköz képességeinek megfelelően azonnal visszaküldi a lejárt TTL mezőjű csomagokat, addig egyes eszközök ezt késleltetve teszik meg.
Az adatsor eredeti hisztogramja nincs bemutatva, mivel egyetlen csúcs látszódna csak az origó környékén, a túlságosan szélsőséges értékek miatt.

\begin{figure}[h!]
	\centering
	\includegraphics[width=0.95\textwidth, keepaspectratio]{figures/link-delay-dist.png}
	\caption{Két végpont közötti késleltetés eloszlása}
	\label{fig:link-delay}
\end{figure}

Az adatsornak ezért a két szélsőséges 5 percentilisét levágva készítettem el a valószínűségi változójának a sűrűségfüggvényét. Annak ellenére, hogy a linkekre becsült késleltetések többnyire 0,1 milliszekundum körül helyezkednek el az átlaguk 6,12 milliszekundum.

\pagebreak

\begin{figure}[h!]
	\centering
	\includegraphics[width=0.95\textwidth, keepaspectratio]{figures/link-length-hist.png}
	\caption{Az útvonalak HOP számának eloszlása}
	\label{fig:link-len}
\end{figure}

Ellenőrzésként a mérésben résztvevő útvonalak késleltetése 93,3 milliszekundum átlagosan, a közbülső linkek száma pedig 15,3 átlagosan, ami igazolja a kimutatást. A linkek számáról részletesebb kimutatást a \ref{fig:link-len} ábra ad, amely a PlanetLab gépei közötti útvonalak HOP számának eloszlását mutatja.

A mérés legfőbb értékét az adja, hogy ezek a mérések nem egy hálózati pontból lettek indítva, hanem közel 200 csomópontból teljes hálót alkotva lettek elindítva a mérések. Ez olyan elemzésekre ad lehetőséget, mint az ú.n Hot-Potato jelenség megfigyelésére, amelyet korábbi kutatások is kimutatták\cite{hot-potato}.

\subsubsection*{A Hot Potato jelenség}

\begin{figure}[!ht]
	\centering
	\includegraphics[width=0.7\textwidth, keepaspectratio]{figures/hot-potato.PNG}
	\caption{Csomagtovábbítási stratégia}
	\label{fig:hot-potato}
\end{figure}

\pagebreak

A \ref{fig:hot-potato} ábrán az látható, ahogy az AS1 ISP azonnal kivezeti a hálózatából a kék és piros küldendő adatcsomagot, hiába lettek azonos cél AS-hez címezve. A csomag továbbítása lefoglalná a saját erőforrásait. Ha más AS felé továbbítja, úgy mások erőforrásait használja. A szerződések ugyanis nem drágítják meg a csomagok továbbításának az árát, ha az távolabbra küldődik.

\begin{figure}[!ht]
	\centering
	\includegraphics[width=0.4\textwidth, keepaspectratio]{figures/asymetric.PNG}
	\caption{Asszimetrikus útvonalválasztás a Hot-Potato jelenség miatt\protect\footnotemark}
	\label{fig:hot-potato}
\end{figure}

\footnotetext{Az ábra a \cite{hot-potato} forrásból lett átvéve}

A mérésekből úgy mutatható ki ez a stratégia, hogy azonos csomópont-pár esetén az egymás felé küldött csomagok nem feltétlen azonos útvonalon haladnak. Ugyanis a küldő ISP más útvonalat választ a hálózatából kifele küldött csomag számára, mint a fogadáskor.

\subsection*{PlanetLab gépek állapota}
A PlanetLab gépek eléréséről (az operációs rendszeréről) és hiba esetén a hibaüzenetek aggregált statisztikája. A következő adatokat tartalmazza:

\begin{itemize}
\item \textbf{erroneous:} Sikertelen csatlakozások száma (online gépek esetén)
\item \textbf{succeed:} Gépek száma, amelyeken sikeres volt a távoli parancsfuttatás (cat /etc/issue)
\item \textbf{online:} A ping parancssal elérhető gépek száma
\item \textbf{offline:} A ping parancssal nem elérhető gépek száma
\item \textbf{outputs:} Az összes különböző kimenet felsorolása a hozzá tartozó előfordulások számával.
\item \textbf{error\_types:} Az összes különböző hibatípus felsorolása a hozzá tartozó előfordulások számával
\item \textbf{ts:} A mérés időpontja
\end{itemize}

\begin{figure}[!ht]
	\centering
	\includegraphics[width=0.9\textwidth, keepaspectratio]{figures/availability.PNG}
	\caption{A PlanetLab hálózat gépeinek elérhetősége egy hetes mintavételen}
	\label{fig:availability}
\end{figure}

Ez az adat kollekció a legfontosabb forrása a mérési rendszer rendelkezésre állásának és a hozzá szükséges PlanetLab hálózat elérhetőségének monitorozásának. A \ref{fig:availability} ábrán látható az online és a succeed adatsorok egy hetes mintavétele. Jól látható, hogy a PlanetLab hálózatának elérhetőségében van egy kevés fluktuáció, maximum 10 gép körüli szórással azonban összességében stabilan elérhetőek.



\subsection*{AS gráf él információi}
A korábban részletesen bemutatott AS gráf ezen kollekció adataiból lett felépítve:

\begin{itemize}
\item \textbf{asn:} A autonóm rendszer azonosító száma
\item \textbf{core\_ips:} Az autonóm rendszeren belül észlelt ip címek
\item \textbf{gateways\_to\_as:} Az autonóm rendszerből másikba vezető kapcsolatok listája. A másik AS-hez irányuló ip cím párok (egyik AS kimeneti címe, másik AS bemeneti címe) listáit is tartalmazza.
\end{itemize}

Ezen adatsor bemutatása nagy részben a következő szekcióban történik majd, mivel az elkészült gráfok adatforrása ez az adat kollekció. Az eredményezett gráf megjelenítéshez viszont nem kapcsolódó kimutatás látható a \ref{fig:as-route-len} ábrán, amely a feltérképezett Autonóm Rendszerek közötti útvonalak hossz eloszlását ábrázolja.

\begin{figure}[h!]
	\centering
	\includegraphics[width=0.9\textwidth, keepaspectratio]{figures/as-hop-hist.png}
	\caption{Az AS gráf csomópontjai között lévő útvonalak hosszának eloszlása}
	\label{fig:as-route-len}
\end{figure}

Az IP linkek és az AS linkek természetes hasonlósága okán a korábbi \ref{fig:link-len} ábrához hasonlít. Az útvonalak hosszának függetlensége miatt a centrális határeloszlás értelmében minél nagyobb a mérés mintavétele, annál jobban hasonlítanak az említett függvények a Normális eloszlásra. Az AS gráf az egy \glqq klaszterhez \grqq tartozó IP csomópontok egybeolvasztása révén jön létre. A kapott hálózat pedig legtöbb tulajdonságát így örökli az eredeti gráfból.


\section{Gráfok ábrázolása}

\begin{figure}[!h]
	\centering
	\includegraphics[width=1\textwidth, keepaspectratio]{figures/graph.png}
	\caption{Egy napi mérés gráfja egy egyetemi cím felé vezető útvonalak IP csomópontjaiból\label{fig:graph}}
\end{figure}

A mérési eredményekből készült gráfok tulajdonságainak intuitív leolvasásához ábrázolások készültek. Mivel egy mérés során több mint 1700 csomópontból álló gráfot kell ábrázolni, ezért ennek kivitelesése kihívást jelent. A \ref{fig:graph} ábrán látható az első sikeres gráf ábrázolás, amelyen már kivehető a gráf szerkezete.
Az ábrán jól láthatóak a középponttól távolodó hosszú fürtök, melyek egy cél IP címhez vezető többnyire független útvonalakat reprezentálja.

\pagebreak


\begin{figure}[!h]
	\centering
	\includegraphics[width=1\textwidth, keepaspectratio]{figures/as-graph2.png}
	\caption{Az IP gráf továbbfejlesztett ábrázolása\label{fig:ip-graph}}
\end{figure}

Az ábrázolás további fejlesztését követően a \ref{fig:ip-graph} ábrán látható kép készült el. Ezen az egyes IP csomópontok sugárirányban a fokszámuk alapján vannak elhelyezve, minél nagyobb fokszámmal rendelkeznek, annál közelebb vannak az origóhoz. Ezen felül klaszterezést követően lettek csoportonként szétszórva a kör peremén. A színezés az egy AS-hez tartozó IP csomópontokat jelöli. A legtöbb esetben a klaszterezés vissza is adta a csomópontok Autonóm Rendszerekhez való rendelését. A kör közepén a korábban említett alacsonyabb Tier szinthez tartozó Autonóm Rendszerek csomópontjai tartoznak.

\pagebreak

A csomópontok koordinátáinak számításához használt funkció a következő volt, amely az R programozási nyelven lett írva:

\begin{lstlisting}[language=R]
my_layout <- function(net){
  # Alap informaciok kinyerese a halozatbol
  nodes = V(net)
  deg = degree(net)
  max_deg = max(deg)
  clust = cluster_louvain(net)
  memb = membership(clust)
  
  # klaszterek szogtartomanyokra osztasa
  group_start = get_group_start_indexes(memb, nodes)
  
  for(n in nodes){
    group = memb[n]
    # origotol valo tavolsag szamitasa
    radius = (max_deg - deg[n] + 1)/(max_deg + 1)
    # klaszterek alapjan valo csoportositas
    angle = step * ( group_start[group] + group_index[group] )
    group_index[group]  = group_index[group] + 1
    
    coord[n, 1] = radius * cos(angle)
    coord[n, 2] = radius * sin(angle)
  }
  
  return(coord)
}
\end{lstlisting}


\pagebreak

\begin{figure}[h!]
	\centering
	\includegraphics[width=0.95\textwidth, keepaspectratio]{figures/as-graph.png}
	\caption{A PlanetLab gépeitől az egyetem felé küldött csomagok által bejárt AS gráf}
	\label{fig:as-graph}
\end{figure}


Az AS szám információ felhasználásával a \ref{fig:as-graph} ábrán látható gráf lett elkészítve. Az ábra adatforrása pár nap folyamatos traceroute mérések eredménye. A PlanetLab csatlakozott gépeiről a BME egyik gépe felé címzett útvonalak lettek aggregálva. A nyilak vastagsága az alapján növekszik, hány különböző IP cím pár közti össze a két AS csomópontot. Ha minden átmenő csomag azonos IP címeken halad keresztül akkor az vékony lesz, míg ha két AS között haladó csomagok több különböző IP cím párokon keresztül utaznak az vastagabban van ábrázolva.

A \ref{fig:as-graph} ábráról leolvasható, mely Autonóm Rendszerek földrajzilag kiterjedtek. Ha ugyanis több különböző IP útvonal köt össze két AS-t akkor azok valószínűleg területileg is több ponton vannak összeköttetésben.

A \ref{fig:as-graph} ábrán az összes útvonal a BME AS2547-es csomópontjába irányul, amely a valóságban kizárólag a AS1955-ös HBONE-AS HUNGARNET-tel van összeköttetésben. Egyes hibás traceroute mérések azonban torzítják ezt. A legnagyobb fokszámú csomópont az AS20965 , amely az AS1955 legfontosabb szomszédja, az elvi 17-ből\footnote{Forrás: Európai Regionális Internetes Nyilvántartó Hivatal honlapja: \href{https://stat.ripe.net/widget/asn-neighbours\#w.resource=1955}{stat.ripe.net}}.

%\begin{figure}[h!]
%	\centering
%	\includegraphics[width=0.95\textwidth, keepaspectratio]{figures/degree.png}
%	\caption{Az IP gráf csomópontjainak fokszáma előfordulási sűrűségük függvényében}
%	\label{fig:ip-degree}
%\end{figure}


\begin{figure}[h!]
	\centering
	\includegraphics[width=0.9\textwidth, keepaspectratio]{figures/ip_map.png}
	\caption{A mérésben résztvevő IP címek geolokációs pozíciójai}
	\label{fig:ip-map}
\end{figure}

Az IP csomópontok geolokációs adatokkal való felgazdagítását követően új ábrázolási mód vált elérhetővé az Internet szerkezetének a megfigyeléséhez. Ennek eredménye a \ref{fig:ip-map} ábrán látható. A mérések reprezentativitását bizonyítja az Internetes penetrációval rendelkező területek nagy részén látható mérésben résztvevő csomópontok sokasága.

\pagebreak

\section{Middlebox felderítés}
A tanszéken végzett Internetes kutatásokhoz hozzáadott értéket jelentett a mérési rendszer. A kutatás középpontjában napjaink egyik Internetes trendje az úgynevezett Middlebox-ok voltak. Ez egy általános fogalom minden olyan hálózati eszközre vonatkozóan, amely beavatkozik és manipulálja a rajta átmenő forgalmat. Ilyen lehet a hagyományos tűzfal vagy terheléselosztó, de manapság újabb és újabb célra használják fel, néha beavatkozva a végpontok közötti kapcsolatba. Ez az Internet alapjait jelentő protokollok működésére akár nem kívánt hatással is lehet, ezért fontos kutatási téma napjainkban.

A kutatáshoz való hozzájárulása a mérési rendszernek az interneten küldött csomagok TTL mezőjének a manipulációját vizsgálta. Egy célgépen a tcpdump nevű alkalmazást volt elindítva olyan beállításokkal, hogy csak a mérésben résztvevő csomagokat vizsgálja. A PlanetLab hálózat elérhető gépein pedig speciálisan elkészített csomagok voltak küldve a célgép felé a maximális 250-es TTL mezővel. A vizsgálat a Middlebox-ok TTL manipulációját figyelte a különböző csomagtípusokra vonatkozóan.

Három csomagtípus manipulációja volt vizsgálva:

\begin{itemize}
\item \textbf{ICMP Ping request} Hagyományos ping üzenet kérési csomagja
\item \textbf{TCP SYN port:22} TCP kapcsolatfelépítési csomag a 22-es portra, amely az ssh kapcsolatok fogadására szolgál
\item \textbf{TCP SYN port:80} TCP kapcsolatfelépítési csomag a 80-as portra, amely a http kapcsolatok fogadására szolgál
\end{itemize}

A felhasznált parancsok a következők voltak:

\begin{lstlisting}[language=bash]
  # Celgepen futtatott parancs ICMP csomagok fogadasara
  sudo tcpdump -vnn -i eth0 icmp[icmptype] == 8 and dst host $ip
  # PlanetLab gepekrol kuldott ICMP csomagok parancsa
  ping -c 1 -t 250 $ip
  
  # A celgepen futtatott parancs TCP SYN csomagok fogadasara
  sudo tcpdump -vnn -i eth0 dst host $ip and "tcp[tcpflags] & (tcp-syn) != 0"
  # A PlanetLab gepeirol kuldott TCP csomagok parancsa
  nmap -v --ttl 250 --max-retries 1 -PS -p 80 $ip
  nmap -v --ttl 250 --max-retries 1 -PS -p 22 $ip
\end{lstlisting}


A mérés egy korábbi tanulmányt\cite{middlebox} vett alapul a felderítéshez. Az Internetes útvonalakon található, ttl mezőt manipuláló Middlebox-ok jelenlétét kimutatta a mérés. A mérés 75 szerverről küldött csomagok alapján lett elkészítve, mindegyik csomagtípusból egyet küldve a célgép felé, kivéve a 80-as portra küldöttek, amelyekből 3 csomag lett küldve gépenként.


\begin{figure}[!ht]
	\centering
	\includegraphics[width=0.3\textwidth, keepaspectratio]{figures/hist-ttl-icmp.png}
	\includegraphics[width=0.3\textwidth, keepaspectratio]{figures/ttl-hist-22.png}
	\includegraphics[width=0.3\textwidth, keepaspectratio]{figures/ttl-hist-port80.png}
	\caption{A fogadott csomagok TTL értékeinek előfordulási sűrűsége}
	\label{fig:ttl-hist}
\end{figure}

A \ref{fig:ttl-hist} ábrán látható, hogy az eredetileg 250-es TTL mezővel küldött csomagok egy része fogadáskor már egy közbülső elem által manipulálva lett. A manipuláció abból következtethető, hogy az útvonalak hosszúsága normális eloszlást követ, ahogy a \ref{fig:hop-hist} ábrán látható, valamint a TTL csökkenésnek függetlennek kellene lennie a csomag típusától.

\begin{figure}[!ht]
	\centering
	\includegraphics[width=0.5\textwidth, keepaspectratio]{figures/hop-hist.png}
	\caption{Az eredeti HOP távolsága a mért útvonalaknak}
	\label{fig:hop-hist}
\end{figure}

Az ICMP csomagok ábráján egyetlen csomag érkezett vélhetően módosított TTL mezővel, későbbi ellenőrzéskor azonban kiderült, egy nem a mérésben résztvevő gép küldte. A méréshez használt csomagszűrő minden Ping requestet átengedett, azonban a mérés ideje alatt a colorado-i egyetem egyik szervere is küldött ilyen csomagot, valószínűleg más Internetes mérés részeként. Ilyen módon kimondhatjuk, hogy az ICMP csomagok TTL mezője nem megy keresztül módosításon. Ezzel szemben a TCP csomagok, amelyeket vizsgáltunk az esetek több mint 10\%-ában módosításon estek keresztül. A 80-as portra küldöttek 11\%-a, míg a 22-es portra küldöttek 15\%-a. Itt megjegyzendő, hogy nem következetes az útvonalakon közbeiktatott manipuláció. Egyes útvonalakon csak a 22-es portra küldött csomagok lettek változtatva, másokon csak a 80-as portra küldöttek és természetesen legtöbb esetben mindkettő. Ezen felül a 80-as portra küldött 3 csomagból, manipuláció esetén a legtöbb esetben egyszerre voltak csökkentett TTL mezőjűek és érintetlenek is.

A vélhető ok, amiért az ICMP csomagokak érintetlenül hagyják a Middlebox-ok, az az ICMP csomagok hálózatdiagnosztikai természetéből adódik. Ezeket valószínűleg a hálózat helyes működésének ellenőrzéséhez is használják az operátorok. A hálózat egészséges működésének érdekében ezért támogatniuk kell az ICMP csomagok továbbításának a protokolloknak megfelelő módját.

Ez a mérés jól mutatja be a mérési rendszerben rejlő potenciált. A PlanetLab hálózatnak köszönhetően, az Internetre reprezentatív eredmények jöhetnek létre a gyorsan összeállítható mérési forgatókönyveknek hála.
%Mérési eredmények (7-10 oldal): kinyert mérések tartalma(delay, jitter, link, AS info, Geolocation), mérések mennyisége, mérések minősége

%
\section{Adatok feldolgozása}
(5 oldal): Hibák kiszűrése, gráf reprezentálás, statisztikák
\citep{traceParse}
%Adatok feldolgozása (5 oldal): Hibák kiszűrése, gráf reprezentálás, statisztikák


%------------------------------------------------
% Együttműködések (6 oldal): Hogyan csatlakoztak be, mit értek el, mivel lett több az egész, együttműködések menete.
\chapter{Együttműködések}
%(6 oldal): Hogyan csatlakoztak be, mit értek el, mivel lett több az egész, együttműködések menete.

A mérési rendszer fejlesztése 2015-ben kezdődött önálló laboratóriumi munkaként. Az első működő verziót követően további témakiírások készültek. Ezek közreműködési lehetőségeket biztosítottak hallgatóknak, hogy saját munkájuk a meglévő mérési rendszert vagy annak eredményeit felhasználják és kiegészítsék azt. A következőkben ezen munkák kerülnek bemutatásra. 

\section{Forgalmi mérési környezet kialakítása az Interneten}
Haja Dávid munkájának rövid bemutatása

\section{AS útvonalváltozások vizsgálata}
Kocsmár Bence munkájának rövid bemutatása

\section{Adatbázis szervezés webes környezetben}
Patkó Ákos munkájának rövid bemutatása

%Vége (4 oldal): Összefoglaló lezárás, függelékek, hivatkozások, stb...
\chapter{Összefoglalás}
1-2 oldal
%Dolgozatom zárásaként összegyűjtöm az elvégzett munkák eredményeit, a főbb szerzett tapasztalatokat, valamint kitekintést nyújtok az elért eredmények felhasználására és a további kutatási lehetőségekre.

\section{Elért eredmények}

%A féléves munkám során továbbfejlesztettem a már meglévő mérési rendszert, amely a PlanetLab hálózatán elérhető gépeken végez internetes méréseket. A mérési folyamat megbízhatóságát nagyban növeltem és az eredmények feldolgozásában is nagy előrehaladást értem el. Az általam fejlesztett rendszerhez könnyen hozzáadható újabb, akár komplexebb mérés. A mérési folyamat teljesen monitorozható az általa szolgáltatott webes honlap segítségével és a lehetséges hibák könnyen kideríthetővé váltak. A kapott mérési eredmények pedig még kezdetlegesen, de megtekinthetőek szintén a webes interfészen.


%--------------------
%A féléves munkám során elmélyítettem az internet felépítéséről és működéséről való ismereteimet, gyakorlatot szereztem a PlanetLab teszthálózat ezres nagyságrendű számítógépeinek elérésében, amelyeken automatizált méréseket végeztem. Tapasztalatokat szereztem nagyobb adatbázisok menedzselésében, valamint a nagy gráfok kezelésébe és ábrázolásába is betekintést nyertem.
%
%Munkám gyümölcseként létrehoztam egy folyamatosan bővülő adatbázist az internetes útvonalakról, amelynek vizsgálatát a félév során csak megkezdeni tudtam.


\section{Továbbhaladási lehetőségek}
%A mérést végző rendszeren már nem szükségesek komoly fejlesztések. A mérési eredményeket feldolgozó eljárásokat fontos a jövőben továbbfejleszteni, valamint a webes interfész felhasználóbaráttá tétele. Az eddig elért munkát Diplomatervként tervezem folytatni, de mint Haja Dávid szakdolgozata is bizonyítja más projektmunkák is könnyen be tudnak csatlakozni a fejlesztésekbe. A bekapcsolódási lehetőségeket konzulensemmel igyekszünk újabb témakiírásokkal támogatni.

%--------------------
%A féléves munka folytatásaként fontosnak tartanám az eddig felhalmozott adatok jobb feldolgozását, és a gráf reprezentációk fejlesztését. Ezeken felül a mérés kibővítésére tennék javaslatokat, mint a mérési időpontok pontosítása (jelenleg a napon belüli bontás nem megfelelő), több cél IP cím hozzáadása, valamint az útvonal szimmetriájának vizsgálata érdekében az útvonalak két oldalról történő mérése.

%%%%%%%%%%%%%%%%%%%%%%%%%%%%%%%%%%%%%%%%%%%%%%%%%
%Önlab1 megjegyzések
%Az eddigi mérési eredményeket úgy gondolom több módon is fel lehetne dolgozni. Egyik magától értetődő cél lehetne az útvonalak késleltetésének, fontosságának (a csomópontok közötti legrövidebb útvonalakban hányszor fordul elő) reprezentálása a világtérképen.
%
%Lehetőség lehetne az internet Autonóm Rendszereinek (AS-eknek) a kapcsolatait vizsgálni. Továbbá az internetes útvonalak további anomáliáinak feltárása is további kutatásra érdemes témának tűnik.

%A mérések menetét is úgy gondolom, hogy tovább lehetne fejleszteni. Jelenleg a mérések időpontjának napon belüli meghatározása problémákba ütközik, de a későbbiekben akár az útvonalak napszaktól függő változásait is nyomon lehetne követni a mérési időpontok pontosításával. További fejlesztési ötlete még a méréseknek, az eddigi két cél IP cím bővítése további célcímekkel, lehetőleg a PlanetLab számítógépeinek címével, mivel így akár az útvonalak szimmetriáját is lehetne vizsgálni, vajon az egyik féltől a másikig ugyanazon az útvonalon haladnak-e a csomagok, mint fordítva?
%Kitekintésként megemlítendő még az eddig használt traceroute parancs lecserélése egy hálózati mérésekben fejlettebb programra.

\newpage
\bibliographystyle{plain}
\bibliography{mybib}

%\listoffigures\addcontentsline{toc}{chapter}{Ábrák jegyzéke}
%\include{appendices}


\chapter*{Függelék}

\section*{Párhuzamos iperf mérés kódja}
\lstinputlisting{example.py}
%
%\section*{AS gráf}
%A gráf csomópontjai az internetet alkotó autonóm rendszerek (AS), a köztük lévő irányított élek a TMIT gépei felé vezető útvonalak, amelyek vastagsága azt mutatja hány különböző ip cím pár lett regisztrálva mint átlépő pont a két AS között.
%
%\bigskip
%\bigskip
%
%
%\includegraphics[width=1\textwidth,keepaspectratio]{figures/as-graph.png}

\end{document}

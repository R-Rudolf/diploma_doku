\chapter{Összefoglalás}
%1-2 oldal
Jelen fejezetben foglalom össze a Diploma eredményeit, annak fő megállapításait, valamint kitekintést nyújtok az elért eredmények felhasználására és a további kutatási lehetőségekre.

\section{Elért eredmények}

Az elkészült automatizált mérési rendszer megbízhatóan monitorozza az Internet egyes régióit, nem csak alapvető méréseket végezve, hanem több lépcsős feldolgozási folyamatok során annak működésére vonatkozó megfigyeléseket is tesz. A kinyert adatok időátfogása és modern tárolási rendszere alkalmassá teszi, hogy Internetes kutatások forrásai legyen. 

A mérési rendszernek kiemelendő a modern és könnyű használata felhős környezetben, valamint a felügyelt, megbízható üzemeltetése.

%A féléves munkám során továbbfejlesztettem a már meglévő mérési rendszert, amely a PlanetLab hálózatán elérhető gépeken végez internetes méréseket. A mérési folyamat megbízhatóságát nagyban növeltem és az eredmények feldolgozásában is nagy előrehaladást értem el. Az általam fejlesztett rendszerhez könnyen hozzáadható újabb, akár komplexebb mérés. A mérési folyamat teljesen monitorozható az általa szolgáltatott webes honlap segítségével és a lehetséges hibák könnyen kideríthetővé váltak. A kapott mérési eredmények pedig még kezdetlegesen, de megtekinthetőek szintén a webes interfészen.


%--------------------
%A féléves munkám során elmélyítettem az internet felépítéséről és működéséről való ismereteimet, gyakorlatot szereztem a PlanetLab teszthálózat ezres nagyságrendű számítógépeinek elérésében, amelyeken automatizált méréseket végeztem. Tapasztalatokat szereztem nagyobb adatbázisok menedzselésében, valamint a nagy gráfok kezelésébe és ábrázolásába is betekintést nyertem.
%
%Munkám gyümölcseként létrehoztam egy folyamatosan bővülő adatbázist az internetes útvonalakról, amelynek vizsgálatát a félév során csak megkezdeni tudtam.


\section{Továbbhaladási lehetőségek}

Az előzőekben részletezett eredményeken alapulva további kutatási irányok határozhatóak meg. Egyik ilyen fő irány lehet a mérési rendszer továbbfejlesztése, valamint az általa begyűjtött mérési adatok további feldolgozása és értelmezése.

A mérési rendszerben rejlő potenciálok kihasználását, annak könnyű felhasználását és megértését célzó nyílt forráskódú projektté alakítása segítené a legjobban. Ennek részeként a forráskód közérthetőségét kell megoldani, valamint részletes dokumentáció és útmutatók kellenek.

A kinyert adatokra alapozva szinte bármilyen kutatási projektet lehet indítani. Az Internet trendjeit követve érdemesebb témának számítana a mobilinternetes hálózati viszonyok vizsgálata, valamint a QoS viszonyok kutatása.

%A mérést végző rendszeren már nem szükségesek komoly fejlesztések. A mérési eredményeket feldolgozó eljárásokat fontos a jövőben továbbfejleszteni, valamint a webes interfész felhasználóbaráttá tétele. Az eddig elért munkát Diplomatervként tervezem folytatni, de mint Haja Dávid szakdolgozata is bizonyítja más projektmunkák is könnyen be tudnak csatlakozni a fejlesztésekbe. A bekapcsolódási lehetőségeket konzulensemmel igyekszünk újabb témakiírásokkal támogatni.

%--------------------
%A féléves munka folytatásaként fontosnak tartanám az eddig felhalmozott adatok jobb feldolgozását, és a gráf reprezentációk fejlesztését. Ezeken felül a mérés kibővítésére tennék javaslatokat, mint a mérési időpontok pontosítása (jelenleg a napon belüli bontás nem megfelelő), több cél IP cím hozzáadása, valamint az útvonal szimmetriájának vizsgálata érdekében az útvonalak két oldalról történő mérése.

%%%%%%%%%%%%%%%%%%%%%%%%%%%%%%%%%%%%%%%%%%%%%%%%%
%Önlab1 megjegyzések
%Az eddigi mérési eredményeket úgy gondolom több módon is fel lehetne dolgozni. Egyik magától értetődő cél lehetne az útvonalak késleltetésének, fontosságának (a csomópontok közötti legrövidebb útvonalakban hányszor fordul elő) reprezentálása a világtérképen.
%
%Lehetőség lehetne az internet Autonóm Rendszereinek (AS-eknek) a kapcsolatait vizsgálni. Továbbá az internetes útvonalak további anomáliáinak feltárása is további kutatásra érdemes témának tűnik.

%A mérések menetét is úgy gondolom, hogy tovább lehetne fejleszteni. Jelenleg a mérések időpontjának napon belüli meghatározása problémákba ütközik, de a későbbiekben akár az útvonalak napszaktól függő változásait is nyomon lehetne követni a mérési időpontok pontosításával. További fejlesztési ötlete még a méréseknek, az eddigi két cél IP cím bővítése további célcímekkel, lehetőleg a PlanetLab számítógépeinek címével, mivel így akár az útvonalak szimmetriáját is lehetne vizsgálni, vajon az egyik féltől a másikig ugyanazon az útvonalon haladnak-e a csomagok, mint fordítva?
%Kitekintésként megemlítendő még az eddig használt traceroute parancs lecserélése egy hálózati mérésekben fejlettebb programra.